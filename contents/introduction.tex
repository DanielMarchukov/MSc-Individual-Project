\section{Introduction}
\subsection{Project Aims, Objectives and Introduction} 
 Argumentation mining is an intensive research field focused on automatically extracting arguments and their interrelationships from text documents. Due to recent technological advancements in other fields such as Artificial Intelligence, Natural Language Processing and Machine Learning, this is becoming more and more possible. Moreover, internet provides unlimited amounts of data to analyze and extract arguments from with accessibility to websites such as Facebook, Twitter, Wikipedia Talks, Reddit and other popular social media pages. 
 
 Unfortunately, this is not easy to do. The definition of argument is hard to express in mathematical or computer science terms. This is because the argument can range several sentences or paragraphs, can contain several arguments within itself and is dependent on meaning, which becomes a natural language processing problem. Context of the argument is also important, because the identification of an argument in a social media site is different from an argument in a law or legal document.

The ability to view and model such arguments provides important insight into how humans reason about different claims, how the claims are supported or disagreed with. This, of course, is a point of interest and connects many other disciplines such as logic, law, philosophy, psychology, social and computer sciences. 

\subsection{Background and Literature Survey} \label{sub:background}
 It gives an overall picture about the work with a clear review of the relevant literature.  The background of the project should be given.  What have been done to deal with the problem should be stated clearly.  The pros and cons of various existing algorithms and approaches should be stated as well.  Differences between your proposed method and the existing ones should be briefly described. 

The following links may help on the literature review: IEEE Xplore digital library: a resource for accessing IEEE published scientific and technical publications. (You must be with King's network to get access to the digital library) ScienceDirect.com: an electronic database offering journal papers not published by IEEE (You must be with King's network to get access to the database)
