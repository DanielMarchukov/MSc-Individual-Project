\section{Introduction} \label{intro}
    The first chapter gives an introduction to the topic and the project. \cref{projoverview} introduces the project and various steps involved in it. Then, \cref{argumentation} introduces the reader to the field of argumentation, with \cref{argumentmining} diving a bit deeper into a more specific subfield, called argument mining. To wrap things up, \cref{aims:objectives} presents the aims and objectives of the project.

    \subsection{Project Overview} \label{projoverview}
        The Project (Argument Mining on Social Media) focuses on several distinct stages and tries to bring them together into one coherent process flow. The overarching expected result is a combination of different subfields (textual data mining, argument identification and interrelationship, argumentation framework build and analysis) into an end to end software. 
     
        The first stage involves gathering raw textual data from social media, namely Twitter, and then cleaning the data. One way of achieving this is stop word elimination (e.g. "and", "or", "the", etc.), because these words do not provide context in isolation. Another technique is lemmatization, which refers to using vocabulary and morphological analysis of words to remove inflectional endings and return only the base or dictionary form of the word (commonly known as lemma). After data is cleaned, it will be ready to go on to the second stage.
     
        The second stage consists of building and training a \gls{rnn} with \gls{lstm} component. This is the actual part where the \gls{te} relationship between two view points will be computed. Data gathered and cleaned in the previous stage will be passed through the \gls{rnn} and the predicted result will be recorded. The computed relationship between two arguments (e.g. sentences, tweets) will directly correlate to how it is going to be depicted in the end result, with positive entailment meaning a support relationship (positive arrow), contradiction entailment - attack relationship (negative arrow) and neutral entailment - no relationship (no arrow) between two nodes in the graph network (which will be built in the following stage) after all data has been processed in this stage. Now that the relationship between arguments is detailed, the third stage can begin.
     
        In the third stage, the main concern is drawing and presenting the results in a format of a graph network. Each vertex will represent an argument, a view point that may be in the form of a tweet consisting of a single or multiple sentences. The edges, on the other hand, will indicate a relationship between the vertices. As such, there will be two types of edges, one that shows that one vertex (argument) supports another, or contradicts (attacks) the other argument. If there is no edge from one vertex to the other, then the two arguments are not related. When the graph is drawn, only the analysis of the graph is left.
     
        Following the newly constructed graph, the analysis of the network can begin. This is the last stage in the project. Analysis of the graph network involves the computation of the most convincing or winning arguments. Alternatively, the aim of the analysis is to take all the arguments and their interrelationships into consideration and produce a result that would show which specific arguments (from the original data set) can be considered as being most likely to represent the consensus of the debate.
     
        In conclusion, the first and second stages cover the fields of data mining and argument interrelationship detection respectively, while the third and fourth stages are involved with transforming the data into an argumentation framework and, consequently, the analysis of it.

    \subsection{Argumentation} \label{argumentation}
        Argumentation is a subject that studies processes involved in reasoning and the structure of reason itself. It is a field that encompasses and spans across multiple other fields, most notably language, philosophy, psychology, logic and has recently transcended into \gls{ai}. This is because \gls{ai} provides the ability to conjoin mental reasoning models and mathematical models for automated reasoning \autocite{Lippi2016ArgumentationMS}. The applications of argument models within computer science and \gls{ai} are vast in possibilities. The ability to view and model such arguments provides important insight into how humans reason about different claims, how the claims are supported or disagreed with. Given an argument model applied to a discussion forum or a debate, this would provide an opportunity to understand more deeply and extract points of view that carry the most weight \autocite{Cocarascu2017MiningBA}. Subsequently, it would be possible to analyse the flow of the discussion and see how the positions taken within the debate change over time as new arguments are introduced. On the other hand, same argument models can also allow us to investigate how a certain perception of something (review, feedback, etc.) can be altered \autocite{ApproxToTruth}. Also, this is a hot topic with regards to ethics within \gls{ai} systems. The decisions that we delegate to \gls{ai} to make, have to be verifiable, thus meaning the system has to be able to show the arguments that it worked with and how the consensus on the decision was made. 
     
        All in all, industries that would benefit would be law, politics and political debates, reviews and marketing, social media platforms to name a few. For example, in politics, and especially political debates, arguments usually go back and forth, and the consensus of the issue of the debate changes throughout the debate itself, making it hard to keep track of, because of required necessity to read and process each argument consecutively. Having an argumentation framework could keep track of the debate in real time, meaning that a user that tuned in in the middle of it, would not need to catch up with the previous arguments, as the argumentation framework would display the consensus at a current point in time. Same principle would apply to social media marketing, by measuring the responses from the users about a certain product or service. Analysis of reviews would also benefit from this, as, for example, the most recent review does not mean that it is the most accurate, or depicts the whole picture of the product and the users usually do not bother reading all the reviews that exist there. Contrastingly, the effect of fake reviews could be analysed, by looking at how the consensus changes with each new review posted, although this would also require the addition of fake review detection element added to the system. As discussed, there are several benefits that encompass different industries, fields and subfields which, rightfully so, make the argumentation a worthwhile endeavour in the research community.
     
    \subsection{Argument Mining} \label{argumentmining}
        \gls{am} is an intensive research field focused on automatically extracting arguments and their interrelationships from text documents. Due to recent technological advancements in other fields such as \gls{ai}, \gls{nlp} and \gls{ml}, this is becoming more and more possible. Moreover, internet provides unlimited amounts of data to analyse and extract arguments from with accessibility to websites such as Facebook, Twitter, Wikipedia Talks, Reddit and other popular social media pages.
         
        Unfortunately, this is not easy to do. The definition of argument is hard to express in mathematical or computer science terms. First of all, there are no limits as to how long an argument should be. An argument, depending on the corpora, can be a part of a sentence or the whole document could be one big argument backed by many smaller arguments within it (e.g. a certain law article is broken down into sections, which are then broken down into paragraphs and so on). Secondly, the nature of the argument has to be understood, to determine whether it is supporting any other argument, or attacking it. An argument is considered to be a supporting argument if it is on point with the view made by the previous original argument. Conversely, going back to the law example, an attack to the argument could be considered in the sense of an exception that is stated within the law, where the law doesn't apply. 
         
        These points coupled together mean that argument classification problem is dependent on meaning of the argument itself, which makes this also a \gls{nlp} problem. Context or topic of the corpora matters a lot as well, because of language stylistic differences. It is no surprise that legal documents entail a more precise syntactical text structure, due to the nature of the industry requiring concrete descriptions to cover and exclude very specific cases of application. On the other hand, texts found in social media are of a more diverse linguistic composition, where messages posted by users can be very conversational or carry a meaning that is only understood by a certain group of people (e.g. abbreviations, specific background required).
    
    \subsection{Project Aims and Objectives} \label{aims:objectives}
        The aim of this project is to gain an up to date knowledge of current trends and advancements within the field of argumentation mining, to create a tool that given an online debate would extract abstract arguments from it, would analyse such arguments and construct an \gls{af} and display the winning or most convincing arguments.
        
        More specifically, the objectives of the project are listed as follows:
        \begin{itemize} 
            \item Provide an introduction to Argumentation and Argument Mining fields.
            \item Explain core technical concepts used in the project.
            \item Discuss the past and ongoing research, its' methods, applications and results.
            \item Mine twitter data and preprocess it, eliminating stop words and performing lemmatization. Save the preprocessed data into .csv format file.
            \item Develop a \gls{lstm} Neural Network, including training, validating and testing its' performance.
            \item Recognise textual entailment relationships between gathered twitter data (arguments).
            \item Transform the labelled arguments into a graph network, including visual representation.
            \item Analyse the resulting graph and compute the grounded semantics.
            \item Analyse project results, neural network performance, compare the acquired result with similar existing works.
            \item Draw conclusions and provide future work suggestions.
        \end{itemize}
    
    The schedule used to reach the objectives listed above is specified in \cref{apx_A}.