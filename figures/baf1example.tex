\begin{figure}[!ht]
    \centering
    \begin{tikzpicture}[->, > = stealth', shorten > = 1pt, auto, node distance = 2cm,
                        thick, main node/.style = {circle, draw, font = \sffamily\Large\bfseries}]
    
        \node[main node] (1) {a};
        \node[main node] (2) [left of=1] {b};
        \node[main node] (3) [below right of=2] {c};
        \node[main node] (4) [right of=1] {d};
        \node[main node] (5) [below right of=4] {e};
        \node[main node] (6) [left of=5] {f};
        \node[main node] (7) [right of=4] {g};
        \node[main node] (8) [right of=7] {h};
        \node[main node] (9) [below left of=8] {i};
        \node[main node] (10) [right of=9] {j};
        \node[main node] (11) [above right of=10] {k};
        
        \path[every node/.style={font=\sffamily\small}]
            (1) edge [right, dotted] node {} (4)
            (2) edge [right] node {} (1)
            (3) edge [right] node {} (4)
            (4) edge [right, dotted] node {} (7)
            (5) edge [above] node {} (7)
            (6) edge [right, dotted] node {} (5)
            (7) edge [right] node {} (8)
            (9) edge [above, dotted] node {} (8)
            (9) edge [right] node {} (10)
            (10) edge [above] node {} (11);
    \end{tikzpicture}
    \caption{Example of Abstract Bipolar Argumentation Framework}
    \label{fig:baf1example}
\end{figure}