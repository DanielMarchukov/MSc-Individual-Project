\section{Technical Background} 
    \subsection{Abstract Argumentation Framework}
        The \textit{argumentation framework} described here is based on \autocite{Dung1995OnTA}. Dung's \gls{af} considers the arguments to be abstract entities without paying any consideration to their internal structure. The way that argument's role is computed is by looking how it relates to other arguments within the argument set.
        
        \begin{definition}
            An \textit{\gls{aaf}} is a tuple $\langle S, \text{ }R \rangle$, where $S$ is a set of \textit{arguments} and $R \subseteq S \times S$ is an \textit{attack relation}.
            \label{definition:definition1}
        \end{definition}
        
        \begin{remark}
            The terms \textit{argumentation framework} and \textit{abstract argumentation framework} will be used interchangeably throughout this document.
            \label{remark:remark1}
        \end{remark}
        
        The \gls{aaf} is represented as a directed graph. Arguments are displayed as nodes and an arrow from node $a$ to node $b$ represents that argument $a$ is attacking argument $b$.
        \begin{exa}
            Assume the following arguments:
            \begin{enumerate}[label=\alph* -]
                \item Today is sunny, we should go out to the park.
                \item I have to study at home for the upcoming exams.
                \item You still have over a month to prepare, so going out for one day won't hurt.
            \end{enumerate}
            \label{exa:example1}
        \end{exa}
        
        The above dialogue example can be represented as an \textit{abstract argumentation framework} $\langle S, \text{ }R \rangle$ where $S = \{a, b, c\}$ and $R = \{(a, b), (b, a), (c, b)\}$, as shown in \autoref{fig:aaf1example} below:
        \begin{figure}[!ht]
    \centering
    \begin{tikzpicture}[->, > = stealth', shorten > = 1pt, auto, node distance = 3cm,
                        thick, main node/.style = {circle, draw, font = \sffamily\Large\bfseries}]
    
      \node[main node] (1) {a};
      \node[main node] (2) [right of=1] {b};
      \node[main node] (3) [right of=2] {c};
    
      \path[every node/.style={font=\sffamily\small}]
        (1) edge [bend right] node [left] {} (2)
        (2) edge [bend right] node {} (1)
        (3) edge [left] node {} (2);
    \end{tikzpicture}
    \caption{Abstract Argumentation Framework for \cref{exa:example1}.}
    \label{fig:aaf1example}
\end{figure}
        
        \gls{af} uses semantics to analyse what sets of nodes (\textit{arguments}) can be reasonably accepted given their \textit{attack relation}. Specifically, for the scope of this work, extension-based semantics are explored. An extension is a set of arguments that are jointly acceptable under special circumstances/properties. An acceptable set of arguments is shown in \autoref{fig:aaf1examplecolorized}, following the \cref{exa:example1}.
        \begin{figure}[!ht]
    \centering
    \begin{tikzpicture}[->, > = stealth', shorten > = 1pt, auto, node distance = 3cm,
                        thick, main node/.style = {circle, draw, font = \sffamily\Large\bfseries}]
    
      \node[main node] (1) [fill=green] {a};
      \node[main node] (2) [right of=1, fill=red] {b};
      \node[main node] (3) [right of=2, fill=green] {c};
    
      \path[every node/.style={font=\sffamily\small}]
        (1) edge [bend right] node [left] {} (2)
        (2) edge [bend right] node {} (1)
        (3) edge [left] node {} (2);
    \end{tikzpicture}
    \caption{Acceptable set of arguments in green.}
    \label{fig:aaf1examplecolorized}
\end{figure}
        
        \subsubsection{Main Concepts}
            Before introducing different extensions, some concepts have to be defined first.
            \begin{definition}
                A set $T \subseteq S$ is \textit{conflict-free} if and only if $ \forall a, b \in T, (a, b) \notin R$
                \label{definition:definition2}
            \end{definition}
            \begin{exa}
                The conflict-free subsets of \autoref{fig:aaf1example} are: $\{\}, \{a\}, \{b\}, \{c\}, \{a, c\}$.
                \label{exa:example2}
            \end{exa}
            
            \begin{definition}
                A set $T \subseteq S$ \textit{defends} $b$ if and only if $ \forall a \in S$ such that $(a, b) \in R$ there exists $\exists c \in T$ such that $(c, a) \in R$.
                \label{definition:definition3}
            \end{definition}
            \begin{exa}
                By looking at \autoref{fig:aaf1example}, the set $\{c\}$ defends $a$ and $\{\}$ defends $c$.
                \label{exa:example3}
            \end{exa}
            
            \begin{definition}
                A set $T \subseteq S$ is \textit{admissible} if and only if $T$ is \textit{conflict-free} and \textit{defends} each argument that is a member of $T$.
                \label{definition:definition4}
            \end{definition}
            \begin{exa}
                By looking at \autoref{fig:aaf1example}, the admissible sets are:  $\{\}, \{a\}, \{c\}, \{a, c\}$.
                \label{exa:example4}
            \end{exa}
            
            \begin{definition}
                $A$ is a \textit{maximal subset} of $S$ in $C$ if and only if $\nexists A' \in C$ such that $A \subsetneq A'$.
                \label{definition:definition5}
            \end{definition}
            \begin{exa}
                Take $S = \{a, b, c\}$ and let $C = \{\{a\}, \{b\}, \{a, b\}, \{b, c\}\}$. The maximal subsets of $C$ in $S$ are $\{a, b\}$ and $\{b, c\}$.
                \label{exa:example5}
            \end{exa}
            
            \begin{definition}
                $A$ is a \textit{minimal} of $S$ in $C$ if and only if $\nexists A' \in C$ such that $A' \subsetneq A$.
                \label{definition:definition6}
            \end{definition}
            \begin{exa}
                Take $S = \{a, b, c\}$ and let $C = \{\{a\}, \{b\}, \{a, b\}, \{b, c\}\}$. The minimal subsets of $C$ in $S$ are $\{a\}$ and $\{b\}$.
                \label{exa:example6}
            \end{exa}
        
        \subsubsection{Semantic Extensions}
            Now moving on to semantic extensions:
            \begin{definition}
                A \textit{complete extension} is an admissible set that includes \textit{all} arguments it defends.
                \label{definition:definition7}
            \end{definition}
            
            \begin{definition}
                A \textit{grounded extension} is the minimal subset of the set of \textit{complete extensions}.
                \label{definition:definition8}
            \end{definition}
            \begin{exa}
                From \autoref{fig:aaf2example} can be seen that complete extensions are: $\{a\}, \{a, c\}, \{a. d\}$. Grounded extensions are: $\{a\}$.
                \label{exa:example8}
            \end{exa}
            \begin{figure}[!ht]
    \centering
    \begin{tikzpicture}[->, > = stealth', shorten > = 1pt, auto, node distance = 2cm,
                        thick, main node/.style = {circle, draw, font = \sffamily\Large\bfseries}]
    
      \node[main node] (1) {a};
      \node[main node] (2) [right of=1] {b};
      \node[main node] (3) [right of=2] {c};
      \node[main node] (4) [above of=3] {d};
      \node[main node] (5) [left of=4] {e};
    
      \path[every node/.style={font=\sffamily\small}]
        (1) edge [right] node {} (2)
        (3) edge [left] node {} (2)
            edge [bend right] node {} (4)
        (4) edge [bend right] node {} (3)
            edge [left] node {} (5)
        (5) edge [loop above] node {} (5);
    \end{tikzpicture}
    \caption{Example of Argumentation Framework}
    \label{fig:aaf2example}
\end{figure}
            \begin{definition}
                A \textit{preferred extension} is the maximal subset of the set of \textit{complete extensions}.
                \label{definition:definition9}
            \end{definition}
            \begin{exa}
                From \autoref{fig:aaf2example} can be seen that complete extensions are: $\{a\}, \{a, c\}, \{a. d\}$. Preferred extensions are: $\{a, c\}, \{a. d\}$.
                \label{exa:example9}
            \end{exa}
        
    \subsection{Bipolar Argumentation Framework}
        The limitation with \gls{aaf}s is that only conflicting relations are captured. The framework has difficulties capturing \textit{support} arguments, that is, when one argument is in agreement of the other. \gls{baf} aims at extending upon the concepts presented by Dung's framework \autocite{Dung1995OnTA}. In \autocite{Cayrol2005OnTA}, the author introduces another dimension through which argument relations can be captured and incorporates it into \gls{aaf}. 
        
        In terms of the framework (graph) itself, nodes still represent the \textit{arguments}, and solid arrows represent \textit{attacks}. \textit{Support} relation is also represented as an arrow, indicating that argument $a$ supports argument $b$. There are no strict requirements for how the arrows have to be drawn within the framework, so in this project a dotted arrow will be used to indicate \textit{support} (see \autoref{fig:baf1example}).
        
        \begin{definition}
            An \textit{Abstract Bipolar Argumentation Framework} is a tuple $\langle S, \text{ }R_{att}, \text{ }R_{sup} \rangle$, where $S$ is a set of \textit{arguments},  $R_{att}$ is a binary \textit{attack relation} and $R_{sup}$ is a binary \textit{support relation}.
            \label{definition:definition10}
        \end{definition}
        \begin{exa}
            Take $T_{i}$, $T_{j} \in S$. $T_{i}R_{att}T_{j}$ (contrastively, $T_{i}R_{sup}T_{j}$) indicates that $T_{i}$ attacks $T_{j}$ (respectively, $T_{i}$ supports $T_{j}$).
            \label{exa:example10}
        \end{exa}
        
        \begin{figure}[!ht]
    \centering
    \begin{tikzpicture}[->, > = stealth', shorten > = 1pt, auto, node distance = 2cm,
                        thick, main node/.style = {circle, draw, font = \sffamily\Large\bfseries}]
    
        \node[main node] (1) {a};
        \node[main node] (2) [left of=1] {b};
        \node[main node] (3) [below right of=2] {c};
        \node[main node] (4) [right of=1] {d};
        \node[main node] (5) [below right of=4] {e};
        \node[main node] (6) [left of=5] {f};
        \node[main node] (7) [right of=4] {g};
        \node[main node] (8) [right of=7] {h};
        \node[main node] (9) [below left of=8] {i};
        \node[main node] (10) [right of=9] {j};
        \node[main node] (11) [above right of=10] {k};
        
        \path[every node/.style={font=\sffamily\small}]
            (1) edge [right, dotted] node {} (4)
            (2) edge [right] node {} (1)
            (3) edge [right] node {} (4)
            (4) edge [right, dotted] node {} (7)
            (5) edge [above] node {} (7)
            (6) edge [right, dotted] node {} (5)
            (7) edge [right] node {} (8)
            (9) edge [above, dotted] node {} (8)
            (9) edge [right] node {} (10)
            (10) edge [above] node {} (11);
    \end{tikzpicture}
    \caption{Example of Abstract Bipolar Argumentation Framework}
    \label{fig:baf1example}
\end{figure}
        
        \subsubsection{Main Concepts}
            The main concepts and definitions of \gls{baf} are presented briefly, based on \autocite{Cayrol2005OnTA}. Firstly, two definitions of attacks are introduced, being \textit{supported} and \textit{indirect}:
            
            \begin{definition}
                Take $A$, $B \subseteq S$. A \textit{supported attack} is a sequence of arguments and relations $A_{1} R_{1} \ldots R_{n - 1} A_{n}$, $n \geq 3$ with $A_{n}$ $=$ $B$, $\forall i$ = $1 \ldots n$ $-$ $2$, $R_{i}$ $=$ $R_{sup}$ and $R_{n - 1}$ $=$ $R_{att}$.
                \label{definition:definition11}
            \end{definition}
            
            \begin{definition}
                Take $A$, $B \subseteq S$. An \textit{indirect attack} is a sequence of arguments and relations $A_{1} R_{1} \ldots R_{n - 1} A_{n}$, $n \geq 3$ with $A_{n}$ $=$ $B$, $\forall i$ = $2 \ldots n$ $-$ $1$, $R_{i}$ $=$ $R_{sup}$ and $R_{1}$ $=$ $R_{att}$.
                \label{definition:definition12}
            \end{definition}
            
            \begin{exa}
                From \autoref{fig:baf1example} it can be seen that sequences $a \rightarrow d \rightarrow g \rightarrow h$ and $e \rightarrow g$ are supported attacks. Subsequently, the sequence $b \rightarrow a \rightarrow d \rightarrow g$ is an indirect attack.
                \label{exa:example11}
            \end{exa}
            
            Next, the \textit{argument-set} versus the \textit{argument} relations are also defined, namely \textit{set-attacks}, \textit{set-supports} and \textit{set-defends}.
            
            \begin{definition}
                Take $T \subseteq S$ and $B \in S$. A \textit{set-attack} from $T$ to $B$ is if and only if there exists a \textit{supported attack} or an \textit{indirect attack} against $B$ from an argument in set $T$.
                \label{definition:definition13}
            \end{definition}
            
            \begin{definition}
                Take $T \subseteq S$ and $B \in S$. A \textit{set-support} relation from set $T$ to argument $B$ is if and only if in the sequence $T_{1} R_{1} \ldots R_{n - 1} T_{n}$, $n \geq 2$, $\forall i$ = $1 \ldots n$ $-$ $1$, $R_{i}$ $=$ $R_{sup}$, $T_{n}$ $=$ $B$, $T_{n - 1} \in S$.
                \label{definition:definition14}
            \end{definition}
            
            \begin{exa}
                From \autoref{fig:baf1example} it can be seen that a set with arguments $\{a, c\}$ set-attacks $h$ and $d$. Also, $\{a, c\}$ set-supports $d$.
                \label{exa:example12}
            \end{exa}
            
            \begin{definition}
                Take $T \subseteq S$ and $A \in S$. A set $T$ \textit{set-defends} $A$ if and only if $\forall B \in S$, if $\{B\}$ \textit{set-attacks} $A$ $\implies$ $\exists C \in S$, $\{C\}$ \textit{set-attacks} $B$.
                \label{definition:definition15}
            \end{definition}
            
            \begin{exa}
                From \autoref{fig:baf1example} it can be seen that a set with arguments $\{b, c, f\}$ and $\{b, c, e\}$ set-defends $h$. Also, $\{b, f\}$ does not set-defend $h$.
                \label{exa:example13}
            \end{exa}
            
            Before moving on to admissibility of sets, conflict-free, safe and closed sets are introduced and expanded upon.
            
            \begin{definition}
                A \textit{conflict-free} set is a .
                \label{definition:definition16}
            \end{definition}
            
            \begin{definition}
                A \textit{safe} set is a .
                \label{definition:definition17}
            \end{definition}
            
            \begin{definition}
                A \textit{closed} set is a .
                \label{definition:definition18}
            \end{definition}
            
        \subsubsection{Semantic Extensions}
            