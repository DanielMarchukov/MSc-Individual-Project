\section{Professional Issues and Ethical Considerations}
    Whenever a project is dealing with personal data, such as messages obtained on social media, there is always a concern for ethical use of such information. Currently there is a lot of discussion surrounding big technology firms, such as Google or Facebook, and their policies with regards to using user data in advertising or other commercial gains. These concerns gave fruit to still quite recent General Data Protection Act being enforced within the European Union.
    
    The project that is being developed here makes use of Twitter data available publicly on the platform. Since the goal of the software is purely research focused, the data gathered from Twitter search functionality is anonymised. Tweets stored inside the software during processing and outside, saved externally, are anonymised, with each tweet being assigned an artificial number instead of using user names. All mentions of other users with "@" tag are removed as well. Additionally, the project code is to be open sourced and is non profit, thus no advertising is to be included inside it or personal data stored.
    
    The process of building argumentation frameworks and analysing them does not require any personal identifiable information, because each framework is looked at in an abstract way, and not with the goal of tracking specific person's argumentation tendencies or political views on different topics. The software is to remain neutral and nonpartisan in its' algorithm implementation throughout software's lifetime.