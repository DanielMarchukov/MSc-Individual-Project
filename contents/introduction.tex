\section{Introduction}
\subsection{Overview} 
 Argumentation is a subject that studies processes involved in reasoning and the structure of reason itself. It is a field that encompasses and spans across multiple other fields, most notably language, philosophy, psychology, logic and has recently transcended into \gls{ai}. This is because \gls{ai} provides the ability to conjoin mental reasoning models and mathematical models for automated reasoning \autocite{Lippi2016ArgumentationMS}. The applications of argument models within computer science and \gls{ai} are vast in possibilities. The ability to view and model such arguments provides important insight into how humans reason about different claims, how the claims are supported or disagreed with. Given an argument model applied to a discussion forum or a debate, this would provide an opportunity to understand more deeply and extract points of view that carry the most weight \autocite{Cocarascu2017MiningBA}. Subsequently, it would be possible to analyse the flow of the discussion and see how the positions taken within the debate change over time as new arguments are introduced. On the other hand, same argument models can also allow us to investigate how a certain perception of something (review, feedback, etc.) can be altered \autocite{ApproxToTruth}. Also, this is a hot topic with regards to ethics within the \gls{ai} system. The decisions that we delegate to \gls{ai} to make, have to be verifiable, thus meaning the system has to be able to show the arguments that it worked with and how the consensus on the decision was made. All in all, industries that would benefit would be law, politics and political debates, reviews and marketing, social media platforms to name a few.
 
\subsection{Argument Mining}
 
 \gls{am} is an intensive research field focused on automatically extracting arguments and their interrelationships from text documents. Due to recent technological advancements in other fields such as \gls{ai}, \gls{nlp} and \gls{ml}, this is becoming more and more possible. Moreover, internet provides unlimited amounts of data to analyse and extract arguments from with accessibility to websites such as Facebook, Twitter, Wikipedia Talks, Reddit and other popular social media pages 
 
 Unfortunately, this is not easy to do. The definition of argument is hard to express in mathematical or computer science terms. First of all, there are no limits as to how long an argument should be. An argument, depending on the corpora, can be a part of a sentence or the whole document could be one big argument backed by many smaller arguments within it (e.g. a certain law article is broken down into sections, which are then broken down into paragraphs and so on). Secondly, the nature of the argument has to be understood, to determine whether it is supporting any other argument, or attacking it. An argument is considered to be a supporting argument if it is on point with the view made by the previous original argument. Conversely, going back to the law example, an attack to the argument could be considered in the sense of an exception that is stated within the law, where the law doesn't apply. 
 
 These points coupled together mean that argument classification problem is dependent on meaning of the argument itself, which makes this also a \gls{nlp} problem. Context or topic of the corpora matters a lot as well, because of language stylistic differences. It is no surprise that legal documents entail a more precise syntactical text structure, due to the nature of the industry requiring concrete descriptions to cover and exclude very specific cases of application. On the other hand, texts found in social media are of a more diverse linguistic composition, where messages posted by users can be very conversational or carry a meaning that is only understood by a certain group of people (e.g. abbreviations, specific background required).

\subsection{Project Aims and Objectives} 
 The aim of this project is to gain an up to date knowledge of current trends and advancements within the field of argumentation mining, to create a tool that given a specific chain of arguments in text form would analyse such arguments and construct an \gls{af} displaying the winning or most convincing arguments, and to contribute to the research community by explaining views and sharing ideas pertaining to the problem at hand.
